\documentclass[a4paper,10pt]{report}


% Title Page
\title{ Course: Distributed Stochastic Simulation
 Report: Assignment \# 5
Traveling Salesman Problem(TSP) Using Simulated Annealing 
}
\author{Ketan C. Maheshwari}


\begin{document}
\maketitle

\begin{abstract}
The famous traveling salesman problem has been attempted by many a mathematicians and computer scientists alike. We attempt to solve the problem using an approach that is used in metallurgy and various other branches of physics like thermodynamics etc..The approach arrives at the best(possibly) solution by randomly chosing a given arrangement and allowing the system to `cool off' to a steady state.While answering each question we refer to the appropriate java program if applicable.
\end{abstract}
\begin{itemize}
\item (Assumption)As no starting point is given for Harry we will consider a random starting point.
\item The equation for the salesman is (n-1)! where n is the total number of cities including the salesman's hometown.
\item To check weather this holds for n=1 to 5 we name the cities a,b,c,d and e. Now lets check the alternatives for each in turn:
for n=1 there is only one city to visit that can be done in only one way so we say that the equation (1-1)! = 0! = 1 holds well.
for n=2 we have cities a and b and we can make a trip to b from a and return back in just one way. So, (2-1)!=1!=1 holds well.
for n=3 we have cities a, b, and c and following trips are possible:\\
        a--b--c\\
        a--c--b\\
which makes totally 2 trips which is same as (3-1)!=2!=2\\
for n=4 we have following trip alteratives:\\
        a--b--c--d\\
        a--b--d--c\\
        a--c--d--b\\
        a--c--b--d\\
        a--d--b--c\\
        a--d--c--b\\
which is six and follows the equation (4-1)!=3!=6\\
Finally, if n=5 we have following trip arrangements:\\
        a--b--c--d--e\\
        a--b--d--c--e\\
        a--b--e--c--d\\
        a--b--c--e--d\\
        a--b--d--e--c\\
        a--b--e--d--c\\
        a--c--b--e--d\\
        .\\
        .\\
        .\\
with this pattern we get 24 different trip alternatives which again follows the equation (5-1)!=4!=24.
\item The program TSP.java/class computes Harry's shortest route using Simulated Annealing Method. With the temperature around 1700, we get the following outputs(for acceptance around 80\%):
Acceptance Probability:80.0008 \%.
The route:
5--3--4--8--10--12--11--16--14--15--7--9--1--13--2--6
The total distance: 15805.50258000084 km.


\item For acceptance around 40\%, we get following output:
Acceptance Probability:43.1207 \%.
The route:
9--1--6--4--14--16--12--10--3--7--8--13--2--5--11--15
The total distance: 17806.522026291193 km.

\item If we use enumeration method then we need to generate each of the (n-1)! combinations of cities and run a comparison for the best possible route which will require at most 15!*(time to compute one combination+time\_to\_comparison) iterations. If we estimate that an average Personal Computer takes 1ms to generate a combination and perform the comparison, it should take  15!*1=1307674368000 ms=414.661 years to find the best possible route. The given method of simulated annealing with approximately 1 million iterations takes on an average approximately 57.507 seconds which is approximately 230 million times more faster.
\item So, according to this analysis, the ideal route for Harry is: Lotus\_eaters--Cicones--Malea--Ithaca2--Circeo--Sirens--Hades--Calypso--Cyclops--Birzebbugia--Aeolia\_island--Laestrygonians--Ithaca--Scylla\_and\_Charibdis--Troy--Eavignan
\item Is Harry Odysseus?.

\end{itemize}
\end{document}          
